\section{\label{sec:model} Model}
An open system of point particles is considered in volume $V\subset\mathbb R^3$ in three space dimensions. The total volume $V$ is partitioned into $N_v$ non-overlapping congruent cubic cells $\Delta_l$, $l\in\{1,...,N_v\}$, each of volume $v$, such that $V$ is the union of all $\Delta_l$'s:
\begin{eqnarray}\label{volume}
	V = \bigcup_{l=1}^{N_v}\Delta_l,\qquad
	\Delta_l \cap \Delta_{l'} = \emptyset, \text{ if } l \neq l',
	\\ \qquad\mbox{and}\quad
	V = v N_v.
\end{eqnarray}
The interaction energy between particles in configuration $\gamma_N = \{{\vb r}_1, ..., {\vb r}_N\}$, where ${\vb r}_i$ is the space coordinate of the $i$-th particle and $N$ is the number of particles in the configuration $\gamma_N$, is defined as follows
\begin{equation*}
	W_{N_v}(\gamma_N) = \frac{1}{2} \sum_{{\vb r}_i,{\vb r}_j \in \gamma} \Phi_{N_v} ({\vb r}_i, {\vb r}_j)
\end{equation*}
where $\Phi_{N_v}$ is given by the Curie-Weiss interaction
% The two-particle interaction energy is defined as
\begin{equation}
	\label{def:curie-weiss-pot}
	\Phi_{N_v}(\vb{r}_i, \vb{r}_j) = -\frac{J_1}{N_v} + J_2\sum_{l=1}^{N_v} \mathbb{I}_{\Delta_l}(\vb{r}_i) \mathbb{I}_{\Delta_l}(\vb{r}_j).
\end{equation}
The first term in $\Phi_{N_v}$ describes a global Curie-Weiss (mean-field-like) attraction for any pair of particles in the system.
The strength of this attraction is controlled by an energy parameter $J_1 > 0$. The second term in $\Phi_{N_v}$ describes a local repulsion between two particles contained within the same cell $\Delta_l$ and is characterized by the parameter $J_2 > 0.$
Here, $\mathbb{I}_{\Delta_l}(\vb{r})$ is the indicator function of a cell $\Delta_l$,
\begin{equation}
	\label{def:I}
	\mathbb{I}_{\Delta_l} (\vb{r}) = \left\{
	\begin{array}{ll}
		1, \quad \vb{r} \in \Delta_l,
		\\
		0, \quad \vb{r} \notin \Delta_l.
	\end{array}
	\right.
\end{equation}
We refer the reader to Refs.~\cite{KKD18,KKD20,RDKPS25arxiv} for a more rigorous definition of the model. The main features of the model are the following:
\begin{itemize}
	\item the model possesses an exact solution in the thermodynamic limit;
	
	\item for sufficiently low temperatures, the model exhibits an infinite sequence of first-order phase transitions between fluid phases of successively increasing density;
	
	\item the phase transition lines between each pair of neighboring phases terminate at respective critical points, and therefore an infinite number of critical points exist in the model.
	
\end{itemize}
The phase diagrams in both the temperature-density and pressure-temperature planes are presented in~\cite{KD22}.

Let us briefly introduce notation to be used in this paper, and then summarize the known results for the model.

The grand partition function is expressed as follows
\begin{equation*}
	\Xi = \sum_{N=0}^{\infty}\frac{\zeta^N}{N!} \int_{V} \dotsc \int_{V} \exp(-\beta W_{N_v}(\gamma_N)) {\rm d} {\vb r}_1 \dotsc {\rm d} {\vb r}_N
\end{equation*}
where $\zeta$ is the activity
\begin{equation*}
	\zeta = \frac{\exp(\beta \mu)}{\Lambda^3},
\end{equation*}
$\beta = k_{\rm B} T$ the inverse temperature, $k_{\rm B}$ the Boltzmann constant, $T$ the temperature, $\mu$ the chemical potential, $\Lambda = (2\pi\beta\hbar^2/m)^{1/2}$ the de Broglie thermal wavelength, $\hbar$ the Planck constant, $m$ the mass of a particle. In the grand partition function the integration goes over all configurations with $N$ particles and then the summation goes over all positive integer values of $N$.

We use the following standard dimensionless variables: the reduced temperature $T^*=k_{\mathrm{B}}T/J_1$, the reduced pressure $P^* = P v/J_1$, the reduced chemical potential $\mu^* = \mu/J_1$, the reduced particle density $\rho^* = \frac{\langle N \rangle}{V} v = \frac{\langle N \rangle}{N_v}$, with $P$ being the pressure, and $\langle N \rangle$ the average number of particles in the system, where the brackets denote a grand-canonical ensemble average. The particle density $\rho = \frac{\langle N \rangle}{V}$ is also used throughout the paper.

For stability of interaction the following condition must hold
$J_2 > J_1.$
For this reason we introduce notation
$a = J_2/J_1.$ and set $a=1.2$ in the current work.

The exact asymptotic expressions for the grand partition function were obtained in~\cite{KKD18,KKD20}. Here, we present the result in the form presented in~\cite{RDKPS25arxiv}:
\begin{equation}
	\Xi \simeq c_{N_v} \exp[N_v E(T^*, \mu^*; \bar{y}_{\rm max})]
\end{equation}
which is asymptotically exact for the limit of $N_v \to \infty$. Here $\bar{y}_{\rm max}$ is a function of both $T^*$ and $\mu^*$ and is found from the condition of maximum for quantity $E$, which is defined as follows
\begin{equation*}
	E(T^*, \mu^*; y) = -\frac{y^2}{2}T^* + \ln K_0(T^*,\mu^*; y),
\end{equation*}
where $K_0(T^*,\mu^*; y)$ is the $0$-th member of the following family of special functions
\begin{eqnarray}\label{def:Kj}
	K_j(T^*,\mu^*;y) = \sum_{n=0}^{\infty} \frac{n^j \left(v^* T^{*3/2} \right)^n}{n!} \exp[\left(y+\frac{\mu^*}{T^*}\right)n - \frac{a}{2T^*}n^2]
\end{eqnarray}
for $j=0,1,2,\ldots$, where $v^* = v/\lambda^3$, $\lambda = (2\pi\hbar^2/mJ_1)^{1/2}$. Quantity $c_{N_v}$ does not play any role in the thermodynamic limit, therefore we do not provide its explicit expression here.

For detailed investigation of the conditions necessary for the global maximum of $E$ and the derivation of relations for $\bar{y}_{\rm max}$, we direct the interested readers to works~\cite{KKD18,KKD20,KD22,RDKPS25arxiv}. In this paper, we employ the methodology developed and the results derived in those works to study structure properties of the model, in particular one- and two-particle densities, and the pair distribution function.
